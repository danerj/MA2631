\documentclass{article}

\usepackage{amsmath}
\usepackage{amssymb}
\usepackage{amsfonts}
\usepackage{fullpage}
\usepackage[shortlabels]{enumitem}


\makeatletter
\def\moverlay{\mathpalette\mov@rlay}
\def\mov@rlay#1#2{\leavevmode\vtop{%
   \baselineskip\z@skip \lineskiplimit-\maxdimen
   \ialign{\hfil$\m@th#1##$\hfil\cr#2\crcr}}}
\newcommand{\charfusion}[3][\mathord]{
    #1{\ifx#1\mathop\vphantom{#2}\fi
        \mathpalette\mov@rlay{#2\cr#3}
      }
    \ifx#1\mathop\expandafter\displaylimits\fi}
\makeatother

\newcommand{\cupdot}{\charfusion[\mathbin]{\cup}{\cdot}}
\newcommand{\bigcupdot}{\charfusion[\mathop]{\bigcup}{\cdot}}


\title{MA 2631 Conference 5}

\setlength\parindent{0pt}
\begin{document}
\maketitle

The exponential random variable with parameter $\lambda >0$:

$$
f(x) = \begin{cases}
\lambda e^{-\lambda x} & x \geq 0 \\
0 & x < 0
\end{cases}
$$

\begin{enumerate}

%%% 1 %%%
\item

Let $X$ be an exponential random variable with parameter $\lambda$. Calculate Var$[X]$ in two ways:

\begin{enumerate}
\item By looking up $E[X]$ in the lecture notes, calculating $E[X^2]$ directly using the definition of expectation, and the formula Var$[X] = E[X^2] - (E[X])^2$;

\item By deriving the moment generating function $M_X(t) = E[e^{tX}]$ (a challenge problem). 
\end{enumerate}

%%% 2 %%%
\item

Suppose that the length of a phone call in minutes is an exponential random variable with parameter $\lambda = \frac{1}{10}$. If someone arrives immediately ahead of you at public telephone booth, find the probability you will have to wait

\begin{enumerate}
\item more than 10 minutes;

\item not more than one standard deviation away from the mean. 

\end{enumerate}

%%% 3 %%%
\item We say that a nonnegative random variable $X$ is memoryless if

$$P[X > s+t \vert X > t] = P[X > s] \quad \text{for all } s, t \geq 0.$$ 

Show that an exponential random variable $X$ with parameter $\lambda$ is memoryless.

%%% 4 %%%

\item
Suppose that the number of miles that a car can run before its battery wears out is exponentially distributed with an average value of $10,000$ miles. If a person desires to take a 5,000 mile trip, what is the probability that he or she will be able to complete the trip without having to replace the car battery?

\end{enumerate}
\end{document}