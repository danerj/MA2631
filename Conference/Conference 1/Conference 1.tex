\documentclass{article}

\usepackage{amsmath}
\usepackage{amssymb}
\usepackage{amsfonts}
\usepackage{fullpage}


\makeatletter
\def\moverlay{\mathpalette\mov@rlay}
\def\mov@rlay#1#2{\leavevmode\vtop{%
   \baselineskip\z@skip \lineskiplimit-\maxdimen
   \ialign{\hfil$\m@th#1##$\hfil\cr#2\crcr}}}
\newcommand{\charfusion}[3][\mathord]{
    #1{\ifx#1\mathop\vphantom{#2}\fi
        \mathpalette\mov@rlay{#2\cr#3}
      }
    \ifx#1\mathop\expandafter\displaylimits\fi}
\makeatother

\newcommand{\cupdot}{\charfusion[\mathbin]{\cup}{\cdot}}
\newcommand{\bigcupdot}{\charfusion[\mathop]{\bigcup}{\cdot}}

\title{MA 2631 Conference 1}

\setlength\parindent{0pt}
\begin{document}
\maketitle

\begin{enumerate}

%%%% 1 %%%%
\item

In how many ways can 3 science fiction books, 4 math books and 1 cooking book
arranged on a bookshelf, if

\begin{enumerate}
\item there are no restrictions on the arrangement?
\item all the science fiction books have to be stored together and also all the math books
have to be stored together?
\item (only) the math books have to be stored together, the rest can be arranged without
restriction?
\end{enumerate}


%%%% 2 %%%%
\item 

Mike has nine friends and wants to throw a party. As his appartment is not big enough
to invite all of them, he decides to invite only six.

\begin{enumerate}
\item How many choices of invitations has he?
\item Two of his friends of his are feuding and will not attend the party together.
Accounting for this fact, how many possibilities has he?
\item Two of his friends are very close and will attend the party only if invited together.
Accounting for this fact, how many possibilities has he?
\end{enumerate}


%%%% 3 %%%%
\item

A coin is tossed repeatedly until the first time ”heads” appears.

\begin{enumerate}
\item Describe mathematically the sample space of this experiment.
\item Describe mathematically the events
$$E = \text{”there are no more than four tails”}$$ 
$$F = \text{”there are at least two tails”}$$
\item Describe mathematically the events $E\cap F$ and $E \cup F^c$
\end{enumerate}


%%%% 4 %%%%
\item

Given a family of events $E_1, E_2, \dots , E_n, \dots$ on some sample space $\Omega$, construct a new
family $F_1, F_2, \dots ,F_n, \dots$ on the same sample space $\Omega$ such that the events $F_i$ are
monotone, ($F_m \subseteq F_n$ for $m \leq n$)and

$$ \bigcup_{k=1}^n F_k = \bigcup_{k=1}^n \text{ for any positive integer } n.$$

{\bf Prove} that the constructed family has the desired properties.

\end{enumerate}

\end{document}