\documentclass{article}

\usepackage{amsmath}
\usepackage{amssymb}
\usepackage{amsfonts}
\usepackage{fullpage}
\usepackage{graphicx}
\usepackage{float}


\makeatletter
\def\moverlay{\mathpalette\mov@rlay}
\def\mov@rlay#1#2{\leavevmode\vtop{%
   \baselineskip\z@skip \lineskiplimit-\maxdimen
   \ialign{\hfil$\m@th#1##$\hfil\cr#2\crcr}}}
\newcommand{\charfusion}[3][\mathord]{
    #1{\ifx#1\mathop\vphantom{#2}\fi
        \mathpalette\mov@rlay{#2\cr#3}
      }
    \ifx#1\mathop\expandafter\displaylimits\fi}
\makeatother

\newcommand{\cupdot}{\charfusion[\mathbin]{\cup}{\cdot}}
\newcommand{\bigcupdot}{\charfusion[\mathop]{\bigcup}{\cdot}}

\title{MA 2631 Assignment 11}
\author{Hubert J. Farnsworth}

\setlength\parindent{0pt}
\begin{document}
\maketitle

\begin{enumerate}

%%%% 1 %%%%
\item

Let $X, Y$ be two random variables with joint cdf $F_{X,Y}$ and marginal cdfs $F_X , F_Y$. For $x,
y \in \mathbb{R}$, express $P[X > x; Y \leq y]$
in terms of $F_{X,Y}$, $F_X$ , and $F_Y$.\\

Answer:
\begin{align*}
&\{Y \leq y\} = \Omega \cap \{Y \leq y\} = (\{X \leq x\} \cupdot \{X > x\}) = (\{X \leq x\} \cap \{Y \leq y\}) \cupdot (\{X > x \} \cap \{Y \leq y\}) \\
&F_Y(y) = P[Y \leq y] = P[X \leq x, Y \leq y] + P[X > x, Y \leq y] = F_{X,Y}(x,y) + P[X > x, Y \leq y] \\
&P[X > x, Y \leq y] = F_Y(y) - F_{X,Y}(x,y)
\end{align*}

%%%% 2 %%%%
\item

Assume that there are 12 balls in an urn, 3 of them red, 4 white and 5 blue. Assume that you draw 2 balls of them, replacing the first ball after noting its color before drawing the second ball. \\

Denote by $X$ the number of drawn red balls and by $Y$ the number of drawn white balls. Calculate the joint probability mass distribution of $X$ and $Y$ as well as their marginal
distributions. Are $X$ and $Y$ independent? \\

Answer: 
\begin{align*}
&p_X(0) = \left(\frac{9}{12}\right)^2 =  \frac{9}{16}, \quad p_X(1) = 2 \cdot \frac{3}{12} \frac{9}{12} =  \frac{6}{16}, \quad p_X(2) = \left( \frac{3}{12}\right)^2 = \frac{1}{16} \\
&p_Y(0) = \left(\frac{8}{12}\right)^2 = \frac{4}{9}, \quad p_Y(1) = 2 \cdot \frac{4}{12}\frac{8}{12}, \quad p_Y(2) = \left(\frac{4}{12}\right)^2 = \frac{1}{9} \\
\end{align*}
\begin{align*}
p_{X,Y}(x,y) \quad &x = 0, \quad x = 1, \quad x = 2 \\
y = 0 \quad &\frac{25}{144} \quad \frac{30}{144} \quad \frac{9}{144} \\
y=1 \quad &\frac{40}{144} \quad \frac{24}{144} \\
y=2 \quad & \frac{16}{144}
\end{align*}

$X$ and $Y$ are \underline{not} independent. For instance, $p_{X,Y}(0,0) = \frac{25}{144} \neq \frac{1}{4} = p_X(0)p_Y(0)$. 

\newpage
%%%% 3 %%%%
\item

Assume that the joint probability mass distribution $p_{X,Y}$ of the random variables $X$ and
$Y$ is given by

\begin{align*}
&p_{X,Y}(1,1) = p_{X,Y}(1,2) = p_{X,Y}(1,3) = \frac{1}{12} \\
&p_{X,Y}(2,1) = p_{X,Y}(2,2) = p_{X,Y}(2,3) = \frac{1}{4}
\end{align*}

\begin{enumerate}
\item
Calculate the marginal probability mass distributions $p_X$ and $p_Y$.

\item
Are $X$ and $Y$ independent?

\item
Calculate the probability mass distribution of the random variable $Z = X/Y$.
\end{enumerate}

Answer:

\begin{enumerate}
\item
$p_X(1) = \frac{1}{4} \quad p_X(2) = \frac{3}{4}$\\
$p_Y(1) = p_Y(1) = p_Y(3) = \frac{1}{3}$

\item
Yes $X$ and $Y$ are independent since $p_X(x)p_Y(y) = p_{X,Y}(x,y)$ for all $(x,y) \in \{1,2\} \times \{1,2,3\}$.

\item
$p_Z(2) = p_{X,Y}(2,1) = \frac{1}{4}$\\
$p_Z(1) = p_{X,Y}(1,1) + p_{X,Y}(2,2) = \frac{1}{3}$\\
$p_Z\left(\frac{2}{3}\right) = p_{X,Y}(2,3) = \frac{1}{4}$\\
$p_Z\left(\frac{1}{2}\right) = p_{X,Y}(1,2) = \frac{1}{12}$\\
$p_Z\left(\frac{1}{3}\right) = p_{X,Y}(1,3) = \frac{1}{12}$
\end{enumerate}

\item
Let $X$ and $Y$ be two independent standard-normal distributed random variables and
define $Z = X^2 + Y^2$. Calculate the cumulative distribution function of $Z$. Which
distribution does $Z$ follow?\\

Answer: For $z<0$, $F_Z(z) = P[Z \leq z] = P[X^2+Y^2 \leq z] = 0$ since $X^2 + Y^2 \geq 0$. \\

For $z \geq 0$,

\begin{align*}
F_Z(z) &= P[X^2 + Y^2 \leq z]
= \iint_{\{x^2+y^2 \leq z\}} f_{X,Y}(x,y) \; dA
= \iint_{\{x^2+y^2 \leq z\}} f_{X}(x)f_{Y}(y) \; dA \\
&= \iint_{\{x^2+y^2 \leq z\}} \frac{1}{\sqrt{2\pi}} e^{-x^2/2} \frac{1}{\sqrt{2\pi}} e^{-y^2/2}\; dA
= \frac{1}{2\pi} \iint_{\{x^2+y^2 \leq z\}} e^{-(x^2+y^2)/2} \; dA \\
&= \int_{0}^{2\pi} \int_{0}^{\sqrt{z}} re^{-r^2/2} \; drd\theta
= \frac{2\pi}{2\pi}\left(-e^{-r^2 / 2}\right)\biggr\vert_{0}^{\sqrt{z}}
= 1-e^{-z/2}\\
&\\
\therefore \quad F_Z(z) &=
\begin{cases}
0 & z < 0 \\
1-e^{-z/2} & z \geq 0
\end{cases}
\end{align*}

Conclude that $Z$ is exponentially distributed with parameter $\lambda = 2$. 

\newpage
%%%% 5 %%%%
\item

Let $X, Y$ be two jointly distributed random variables with joint density

$$
f_{X,Y}(x,y)=
\begin{cases}
cxy & 0 \leq x \leq 1 \text{ and } 0 \leq y \leq 1 \\
0 & \text{else}
\end{cases}
$$

\begin{enumerate}
\item
Determine the value of the constant $c$. 

\item
Are $X$ and $Y$ independent?

\item
Calculate $E[X]$.
\end{enumerate}

Answer: 

\begin{enumerate}
\item
$
\frac{1}{c} = \int_0^1 \int_0^1 xy \; dx dy = \left(\frac{x^2}{2}\right)\biggr\vert_0^1
\left(\frac{y^2}{2}\right)\biggr\vert_0^1
= \frac{1}{4}
\implies \boxed{c = 4}
$
\item
$
f_X(x) = 
\begin{cases}
\int_0^1 4xy \; dy & 0 \leq x \leq 1 \\
0 & \text{else}
\end{cases}
=
\begin{cases}
2x & 0 \leq x \leq 1 \\
0 & \text{else}
\end{cases}
$\\
$
f_Y(y) = 
\begin{cases}
\int_0^1 4xy \; dx & 0 \leq y \leq 1 \\
0 & \text{else}
\end{cases}
=
\begin{cases}
2y & 0 \leq y \leq 1 \\
0 & \text{else}
\end{cases}
$

Since $f_{X,Y}(x,y) = 4xy = (2x)(2y) = f_X(x)f_Y(y)$ for $(x,y) \in [0,1]^2$ and $f_{X,Y}(x,y) = 0 = f_X(x)f_Y(y)$ for $(x,y) \notin [0,1]^2$, $f_{X,Y}(x,y) = f_X(x)f_Y(y)$ for all $(x,y) \in \mathbb{R}^2$. Conclude that $X$ and $Y$ are independent. 

\item
$E[X] = \int_0^1 2x^2 \; dx = \frac{2}{3}$
\end{enumerate}

%%%% 6 %%%%
\item
Let $X_1, \dots ,X_n$ be iid random variables with
density $f$ and cumulative distribution function $F$. Calculate in terms of $f$ and $F$ the density and cumulative distribution functions of the random variables
\begin{enumerate}
\item
$Y = \min\{X_1,\dots, X_n\}$,

\item
$Z = \max\{X_1,\dots , X_n\} \;.$
\end{enumerate}

Answer:

\begin{enumerate}
\item
\begin{align*}
F_Y(y) &= P[Y \leq y]
= P[\min\{X_1, \dots , X_n\} \leq y] 
= P[\{X_1 \leq y\} \cup \dots \cup \{X_n \leq y\}] \\
&= 1- P[X_1 > y, \dots , X_n > y]
= 1- P[X_1 > y] P[X_2 > y] \dots  P[X_n > y] \\
&= 1- (1-F(y))(1-F(y))\dots (1-F(y)) \\
&= \boxed{1 - (1-F(y))^n} \\
f_Y(y) &= F_Y'(y)  = \boxed{n(1-F(y))^{n-1}f(y)}
\end{align*}
\item
\begin{align*}
F_Z(z) &= P[Z \leq z] = P[\max\{X_1, \dots , X_n\} \leq z]\\
&= P[\{X_1 \leq z\} \cap \dots \cap \{X_n \leq z\}]
= P[X_1 \leq z, X_2 \leq z, \dots , X_n \leq z]\\
&= \boxed{(F(z))^n}\\
f_Z(z) &= F_Z'(z) = \boxed{n(F(z))^{n-1}f(z)}
\end{align*}
\end{enumerate}

\end{enumerate}

\end{document}