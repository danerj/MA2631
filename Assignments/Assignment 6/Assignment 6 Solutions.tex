\documentclass{article}

\usepackage{amsmath}
\usepackage{amssymb}
\usepackage{amsfonts}
\usepackage{fullpage}
\usepackage{graphicx}
\usepackage{float}


\makeatletter
\def\moverlay{\mathpalette\mov@rlay}
\def\mov@rlay#1#2{\leavevmode\vtop{%
   \baselineskip\z@skip \lineskiplimit-\maxdimen
   \ialign{\hfil$\m@th#1##$\hfil\cr#2\crcr}}}
\newcommand{\charfusion}[3][\mathord]{
    #1{\ifx#1\mathop\vphantom{#2}\fi
        \mathpalette\mov@rlay{#2\cr#3}
      }
    \ifx#1\mathop\expandafter\displaylimits\fi}
\makeatother

\newcommand{\cupdot}{\charfusion[\mathbin]{\cup}{\cdot}}
\newcommand{\bigcupdot}{\charfusion[\mathop]{\bigcup}{\cdot}}

\title{MA 2631 Assignment 6}
\author{Hubert J. Farnsworth}

\setlength\parindent{0pt}
\begin{document}
\maketitle

\begin{enumerate}

%%%% 1 %%%%
\item

The picture on the last page (taken from
https://www.mass.gov/info-details/covid-19-response-reporting) shows that
in early September 2021, the percentage of positive Covid-tests were rising for both,
higher education tests and non-higher education tests, though the overall positivity rate
was actually slightly declining.\\

At first sight this might look paradoxical, and you might even think that there might be
an error. However, you should be able to explain this phenomenon with your knowledge
about conditional probabilities and related theorems. Please do so.\\

Answer: Let $HE$ be the event that the test was a higher education test and $NHE$ the event that test was a non-higher education test. Let $P$ be the event that the test is positive.

$$
\mathbb{P}[P] = \mathbb{P}[P | HE] \mathbb{P}[HE] + \mathbb{P}[P | NHE] \mathbb{P}[NHE]
$$

While both $mathbb{P}[P | HE]$ and $\mathbb{P}[P | NHE]$ may be increasing, the fact that $\mathbb{P}[HE]$ is increasing at the expense of $\mathbb{P}[NHE]$ (as more tests are beginning to be performed at higher education institutions) and that $\mathbb{P}[P|HE]$ is much smaller than $P[P | NHE]$ causes $\mathbb{P}[P]$ to decrease overall. 

%%%% 2 %%%%
\item

Consider the random variable X with the probability mass distribution

$$
P[X = 1] = 0.3, \quad P[X = 4] = 0.25, \quad P[X = 7] = 0.4, \quad P[X = 10] = 0.05.
$$

Calculate the expected value of $X$ and $Y$ with $Y = 3X + 2$.

Answer:

$$
E[X] = 0.3 \cdot 1 + 0.25 \cdot 4 + 0.4 \cdot 7 + 0.05 \cdot 10 = 4.6
$$
$$
E[Y] = E[3X + 2] = 3E[X] + 2 = 15.8
$$

%%%% 3 %%%%
\item

Let $X$ be a random variable describing the number of failures before the first success of
an independently repeated experiment with success probability $p = \frac{3}{4}$.

\begin{enumerate}
\item Calculate the probability that there are not more than two failures before the first success.

\item Calculate $E[2^X]$.

\end{enumerate}

Answer: 

\begin{enumerate}
\item $P[X = 0] + P[X = 1] + P[X = 2] = \frac{3}{4}(1 + \frac{1}{4} + \frac{1}{16}) = \frac{63}{64} \approx 98.44\%$.

\item $E[2^X] = \sum_{k=0}^\infty 2^k P[X = k] 
= \sum_{k=0}^\infty 2^k \left(\frac{1}{4}\right)^k\frac{3}{4}
= \sum_{k=0}^\infty \frac{3}{4}\left(\frac{1}{2}\right)^k = \frac{3}{4} \cdot 2 = \frac{3}{2}$.

\end{enumerate}

%%%% 4 %%%%
\item 

Let $X$ be a random variable taking values in $\mathbb{N}$. Show that $E[X] = \sum_{n=1}^\infty P[X \geq n]$. \\

Answer: 

$$
\sum_{n=1}^\infty P[X \geq n]
= \sum_{n=1}^\infty \sum_{k = n}^\infty P[X = k] 
= \sum_{k=1}^\infty \sum_{n=1}^k P[X = k]
= \sum_{k=1}^\infty kP[X=k]
= E[X].
$$




\end{enumerate}

\end{document}