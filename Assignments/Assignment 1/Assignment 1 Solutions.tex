\documentclass{article}

\usepackage{amsmath}
\usepackage{amssymb}
\usepackage{amsfonts}
\usepackage{fullpage}


\makeatletter
\def\moverlay{\mathpalette\mov@rlay}
\def\mov@rlay#1#2{\leavevmode\vtop{%
   \baselineskip\z@skip \lineskiplimit-\maxdimen
   \ialign{\hfil$\m@th#1##$\hfil\cr#2\crcr}}}
\newcommand{\charfusion}[3][\mathord]{
    #1{\ifx#1\mathop\vphantom{#2}\fi
        \mathpalette\mov@rlay{#2\cr#3}
      }
    \ifx#1\mathop\expandafter\displaylimits\fi}
\makeatother

\newcommand{\cupdot}{\charfusion[\mathbin]{\cup}{\cdot}}
\newcommand{\bigcupdot}{\charfusion[\mathop]{\bigcup}{\cdot}}

\title{MA 2631 Assignment 1}
\author{Hubert J. Farnsworth}

\setlength\parindent{0pt}
\begin{document}
\maketitle

\begin{enumerate}

%%%% 1 %%%%

\item Twenty workers are to be assigned to twenty different jobs. How many different assignments are possible?\\

Answer: $20!$

%%%% 2 %%%%

\item A tourist wants to visit six out of thirteen American cities; seven of them are on the East Coast, three on the West Coast and three in the middle of the country. In how
many ways can she do that if

\begin{enumerate}
\item the order of the visits does not play a role?
\item the order of the cities is important?
\item the order is not important, but she wants to
visit at least three cities on the East Coast and at
least two on the West Coast?
\end{enumerate}

Answer:

\begin{enumerate}
\item Since order does not matter, there are $\binom{13}{6} = 1716$ ways to choose 6 of the 13 cities to visit. 
	
\item Since order does matter, there are 13 choices for the first city, 12 choices for the second city, $\dots$, 8 choices for the sixth city. There are $13! / (13-6)! = 13! / 7! = 1, 235, 520$ ways to do it. 
	
\item Since she visits a total of 6 cities, the possibilities under the given constraints are:

\begin{enumerate}
\item 3 East Coast cities, 2 West Coast cities, 1 Middle city.
\item 4 East Coast cities, 2 West Coast cities, 0 Middle of country cities.
\item 3 East Coast cities, 3 West Coast cities, 0 Middle of country cities.
\end{enumerate}

To find the total number of ways she can plan her trip, add up the number of ways she can do each of i, ii, and iii. 

$$
\binom{7}{3} \binom{3}{2} \binom{3}{1}
+ \binom{7}{4} \binom{3}{2} \binom{3}{0}
+ \binom{7}{3} \binom{3}{3} \binom{3}{0}
= 455.
$$
\end{enumerate}

%%%% 3 %%%%

\item
How many words can you build from the letters ARRANGE if

\begin{enumerate}
\item you have to use all the letters?
\item you do not have to use all the letters (but every word has at least one letter)
\end{enumerate}

Answer:

\begin{enumerate}
\item There are 7! ways to arrange the 7 letters in ARRANGE. But then since the letter A appears twice and the letter R appears twice, we must divide by the number of ways we can arrange duplicates. This gives a total of $7!/(2!2!) = 7!/4 = 1260$ ways to do it. 
\item Consider each of the possible word lengths and then the subcases for each word length. Generally there are three subcases for each word length: \\

Subcase 1. All letters in the word are different (no repeated letters). \\

Subcase 2. Two of the letters in the word are repeats (either both A's appear or both R's appear but not all four of A,A,R,R).\\

Subcase 3. Four of the letters in the word are repeats (both A's appear and both R's appear in the word). \\

Note that note all subcases may be possible for each word length. For example, for a 6 or 7 letter word it's not possible to have no repeated letters and for a 3 letter word it's not possible to have four letters that are repeats, etc.

\begin{enumerate}
\item 1 letter: 5 words possible.

\begin{itemize}
\item The only possible subcase is that all letters are different (there's only 1 letter of course!) and there are 5 different letters to choose from so $\binom{5}{1} = 5$ words possible. 
\end{itemize}

\item 2 letters: 22 words possible.

\begin{itemize}
\item If the letters are all different there are $5\cdot 4 = 20$ words possible 
\item If both letters are the same then either the word is AA or RR: 2 words possible.
\end{itemize}

\item 3 letters: 84 words possible.

\begin{itemize}
\item If all letters are different there are $5 \cdot 4 \cdot 3 = 60$ words possible.
\item If two letters are the same then there are 4 choices remaining for the third letter. Once the letters have been chosen there are 3 ways to order the letters but then we must divide by the number of ways to order the repeats so in total there are $\binom{2}{1}\binom{4}{1}3! / 2!= 24$ words possible.
\end{itemize}

\item 4 letters: 270 words possible. 

\begin{itemize}
\item If the letters are all different there are $5\cdot 4 \cdot 3 \cdot 2 \cdot = 120$ words possible.

\item If two of the letters are the same then there are $\binom{2}{1}\binom{4}{2} 4! / 2! = 144$ words possible.

\item If four of the letters are the same there are $\binom{2}{2}\binom{3}{0}4!/(2!2!) = 6$ words possible. 
\end{itemize}

\item 5 letters: 690 words possible. 

\begin{itemize}
\item If the letters are all different there are $5! = 120$ words possible.

\item If two of the letters are the same there are $\binom{2}{1}\binom{4}{3}5!/2! = 480$ words possible. 

\item If four of the letters are the same there are $\binom{2}{2}\binom{3}{1}5!/(2!2!) = 90$ words possible.
\end{itemize}

\item 6 letters: 1260 words possible.
\begin{itemize}
\item If two of the letters are the same there are $\binom{2}{1}\binom{4}{4}6! / 2! = 720$ words possible.
\item If four of the letters are the same there are $\binom{2}{2}\binom{3}{2}6!/(2!2!) = 540$ words possible. 
\end{itemize} 

\item 7 letters: 1260 words possible (by part a).
\end{enumerate}

{\bf Final result:} adding up the numbers from (i) through (vii) conclude that 3591 words can be built. 
\end{enumerate} 

%%%% 4 %%%%

\item Two experiments are to be performed. The first one can result in $r$ different outcomes.
If the first experiment results in outcome $j$, then the second experiment can result in $n_j$ possible outcomes, $j = 1, . . . , r$. What is the total number of possibles outcomes of the two experiments?

Answer: Add up the number of outcomes in the second experiment for each of the outcomes in the first experiment. There are a total of $n_1 + n_2 + \dots + n_r$ possible outcomes of the two experiments. 

%%%% 5 %%%%

\item Prove by induction that for all positive integers n that

$$
\sum_{k=0}^n \binom{n}{k} = 2^n
$$

Answer: For the base case $n=1$, $\sum_{k=0}^n \binom{n}{k} =\binom{1}{0} + \binom{1}{1}  = 1 + 1 = 2 = 2^n$.

For the inductive step assume that the formula holds for some positive integer $n$.

\begin{align*}
\sum_{k=0}^{n+1} \binom{n+1}{k}
&= \binom{n+1}{0} + \sum_{k=1}^{n} \binom{n+1}{k} + \binom{n+1}{n+1} \\
&= 1 + \sum_{k=1}^{n} \left(\binom{n}{k-1} + \binom{n}{k}\right) + 1 \\
&= \left(1 + \sum_{k=1}^n \binom{n}{k-1} \right) + \left(1 + \sum_{k=1}^n \binom{n}{k} \right) \\
&= \left(1 + \sum_{j=0}^{n-1} \binom{n}{j} \right) + \left(\binom{n}{0} + \sum_{k=1}^n \binom{n}{k} \right) \\
&= \left(\binom{n}{n}+ \sum_{j=0}^{n-1} \binom{n}{j} \right) + \sum_{k=0}^n \binom{n}{k} \\
&= \sum_{j=0}^n \binom{n}{j} +  \sum_{k=0}^n \binom{n}{k} \\
&= 2^n + 2^n = 2^{n+1}.
\end{align*}

%%%% 6 %%%%

\item A soccer coach has 2 goalkeepers and 15 field players at his disposition: 5 defenders, 7
midfielders and 3 forwards.

\begin{enumerate}
\item How many different soccer teams (consisting of 1 goalkeeper and 10 field players) can he build up from these players?
\item If he wants additionally that there are at least 3 defenders, at least 4 midfielders and
at least 2 forwards in his team, how many teams he can form under this restriction?
\end{enumerate}  

Answer:

\begin{enumerate}
\item He must choose 1 of the 2 goalkeepers and 10 of the 15 field players. The order in which he chooses does not matter, so he has $\binom{2}{1} \binom{15}{10} = 6006$ ways to do it. 
\item He must still choose a goalkeeper and there are $\binom{2}{1} = 2$ ways to do this. So find out the number of ways to choose 10 field players under the given constraints and multiply that number by 2. His options are:

\begin{enumerate}
\item 4 defenders, 4 midfielders, 2 forwards,
\item 3 defenders, 5 midfielders, 2 forwards,
\item 3 defenders, 4 midfielders, 3 forwards.
\end{enumerate}

The total number of different soccer teams he can build is

$$\binom{2}{1}\left(\binom{5}{4}\binom{7}{4}\binom{3}{2} + \binom{5}{3}\binom{7}{5}\binom{3}{2} + \binom{5}{3}\binom{7}{4}\binom{3}{3}\right) = 3010.
$$
\end{enumerate}

\end{enumerate}



\end{document}