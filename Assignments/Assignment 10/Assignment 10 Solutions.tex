\documentclass{article}

\usepackage{amsmath}
\usepackage{amssymb}
\usepackage{amsfonts}
\usepackage{fullpage}
\usepackage{graphicx}
\usepackage{float}


\makeatletter
\def\moverlay{\mathpalette\mov@rlay}
\def\mov@rlay#1#2{\leavevmode\vtop{%
   \baselineskip\z@skip \lineskiplimit-\maxdimen
   \ialign{\hfil$\m@th#1##$\hfil\cr#2\crcr}}}
\newcommand{\charfusion}[3][\mathord]{
    #1{\ifx#1\mathop\vphantom{#2}\fi
        \mathpalette\mov@rlay{#2\cr#3}
      }
    \ifx#1\mathop\expandafter\displaylimits\fi}
\makeatother

\newcommand{\cupdot}{\charfusion[\mathbin]{\cup}{\cdot}}
\newcommand{\bigcupdot}{\charfusion[\mathop]{\bigcup}{\cdot}}

\title{MA 2631 Assignment 10}
\author{Hubert J. Farnsworth}

\setlength\parindent{0pt}
\begin{document}
\maketitle

\begin{enumerate}

%%%% 1 %%%%
\item

Assume that $X$ is a normally distributed random variable with mean $\mu$ and variance $\sigma^2$.
Compute the probabilities that $X$ is not more than one, two and three standard
deviations away from the mean, i.e. $P[|X -\mu| \leq\sigma], P[|X -\mu| \leq 2\sigma]$ and
$P[|X -\mu| \leq 3\sigma]$.\\

Answer:

\begin{align*}
P[|X-\mu| \leq k \sigma] &= P[\mu -k\sigma \leq X  \leq \mu + k\sigma] \\
&= \int_{\mu - k\sigma}^{\mu + k\sigma} \frac{1}{\sqrt{2\pi}} e^{-(x-\mu)^2 / (2\sigma^2)} \, dx \\
&\approx \begin{cases}
68.27 \% & k = 1 \\
95.45 \% & k = 2 \\
99.73 \% & k = 3
\end{cases}
\end{align*}

%%%% 2 %%%%
\item 

Let $X$ be a standard normal distributed random variable. Find the value of $\beta \in \mathbb{R}$  such that $P[X^2 < \beta]= 0.5$. \\

Answer:
$$
0.5 = P[X^2 < \beta] = P[-\sqrt{\beta} < X < \sqrt{\beta}].
$$

Look in table of cdf values for a standard normal random variable to find $\Phi(0.75)$ so that there is $25\%$ on each side. It says $0.67 < x < 0.68$, so take $x \approx 0.675$. That is, $0.5 \approx P[-0.675 < X < 0.675]$.\\

Pick $\beta = 0.675^2 = \boxed{0.455625}$ to satisfy $P[X^2 < \beta] \approx 0.5$.

\newpage
%%%% 3 %%%%
\item

Let $X \sim \mathcal{N}(0,1)$ be a standard normal distributed random variable. Calculate the
moment generating function $m_X(\lambda) = E[e^{\lambda X}]$ for $\lambda \in \mathbb{R}$. Use the moment generating function to calculate the mean and variance of $X$. \\

Answer:
\begin{align*}
m_X(\lambda) &= E[e^{\lambda X}]
= \int_{-\infty}^\infty e^{\lambda x} \frac{1}{\sqrt{2\pi}} e^{-x^2 / 2} \, dx \\
&= \frac{1}{\sqrt{2\pi}} \int _{-\infty}^\infty e^{-(x^2 - 2\lambda x) /2} \, dx \\
&= \frac{e^{\lambda^2 / 2}}{\sqrt{2\pi}} \int_{-\infty}^\infty e^{-(x - \lambda)^2 / 2} \, dx \\
&=  \frac{e^{\lambda^2 / 2}}{\sqrt{2\pi}} \sqrt{2\pi}  \\
&= \boxed{e^{\lambda^2 / 2}} \\
& \\
m_X'(\lambda) &= \lambda e^{\lambda^2 /2} \implies \mu = m_X'(0) = 0 \\
m_X''(\lambda) &=  \lambda^2e^{\lambda^2 /2} +  e^{\lambda^2 / 2}\ \implies \sigma^2= m_X''(0) - \mu = 1 \\
\end{align*}

%%% 4 %%%
\item
An Airline sold 560 tickets for an Airbus 380 flight (capacity: 555 seats) in the
assumption that not all passengers that bought a ticket will arrive for the flight.
Assume that the probability that a passenger will not shop up for the flight is 1\%,
independently for all passengers. How likely is it that more passengers show
up for the flight than there are seats are available? Calculate this probability by using:

\begin{enumerate}
\item
a binomial distribution for the number of passengers that showed up for the flight;
\item
a normal approximation.
\end{enumerate}

Answer:

\begin{enumerate}
\item
$\sum_{k=556}^{560} \binom{560}{k}(.99)^k(0.01)^{560-k} \approx 34.09\%$

\item
Choose one of the following three approximations. The third approximation includes the continuity correction so it is the most accurate of the three approximations. 
\begin{align*}
P[X \geq 556] &= P\left[\frac{X - \mu}{\sigma} \geq \frac{556 - \mu}{\sigma}\right]
= 1-P\left[\frac{X - \mu}{\sigma} \leq \frac{1.6}{\sqrt{5.544}}\right] \\
&\approx 1 - \Phi\left(\frac{1.6}{\sqrt{5.544}}\right) \approx 1 - \Phi(0.680) = 1- 0.7517 \approx 24.83 \% \\
P[X > 555] &= P\left[\frac{X - \mu}{\sigma} > \frac{554 - \mu}{\sigma}\right]
= 1-P\left[\frac{X - \mu}{\sigma} \leq \frac{0.6}{\sqrt{5.544}}\right] \\
&\approx 1 - \Phi\left(\frac{0.6}{\sqrt{5.544}}\right) \approx 1 - \Phi(.255) = 1- 0.6007  = 39.03 \% \\
P[X > 555.5] &= P\left[\frac{X - \mu}{\sigma} > \frac{555.5 - \mu}{\sigma}\right]
= 1-P\left[\frac{X - \mu}{\sigma} \leq \frac{1.1}{\sqrt{5.544}}\right] \\
&\approx 1 - \Phi\left(\frac{1.1}{\sqrt{5.544}}\right) \approx 1 - \Phi(.467) = 1- 0.68 = 32\% \\
\end{align*}
\end{enumerate}

\newpage
%%%% 5 %%%%
\item

Let $X$ by a standard normal distributed random variable. Calculate $E[X^n]$ for an
arbitrary non-negative integer $n$.\\

Answer:

Suppose $n$ is odd.

\begin{align*}
E[X^n] &= \int_{-\infty}^\infty x^n f_X(x) dx 
= \int_{-\infty}^0 x^n f_X(x) dx + \int_0^\infty x^n f_X(x) dx \\
&= \int_{\infty}^0 (-y)^n f_X(-y) (-dy) + \int_{0}^\infty x^n f_X(x) dx  \\
&= -\int_{0}^\infty y^nf_X(y) dy + \int_0^\infty x^n f_X(x) dx = 0.
\end{align*}

Suppose $n$ is even and $n\geq 2$ (we already know $E[X^0] = E[1] = 1$). Let $a = 1/\sqrt{2\pi}$. 

\begin{align*}
E[X^n] &= a\int_{-\infty}^\infty x^ne^{-x^2 /2} dx 
= a\int_{-\infty}^\infty x^{n-1}\left(x e^{-x^2 /2}\right) dx \\
&= a\left[-x^{n-1}e^{-x^2/2} \biggr\vert + (n-1)\int_{-\infty}^\infty x^{n-2}e^{-x^2/2} dx \right] \\
&= (n-1)\int_{-\infty}^\infty x^{n-2}a e^{-x^2/2} dx \\
&= (n-1)E[X^{n-2}]. \\
& \\
E[X^0] &= 1 \\
E[X^2] &= (2-1)E[X^0] = 1 \\
E[X^4] &= (4-1)E[X^2] = 3 \cdot 1 \\
E[X^6] &= (6-1)E[X^4] = 5 \cdot 3 \cdot 1 \\
& \vdots \\
E[X^n] &= (n-1) \cdot (n-3) \cdot  \dots \cdot 3 \cdot 1. 
\end{align*}

Prove this last claim by induction. The base case has already been shown. Assume the identity holds for an even integer $n \geq 2$. Replacing $n$ with $n+2$ in the integration above, $E[X^{n+2}] = (n+1)E[X^n] = (n+1)\cdot (n-1) \cdot \dots \cdot 1$. Using the formula $1 \cdot 3 \cdot \dots \cdot k = (2k)! / (2^k k!)$ for odd $k$ conclude:

$$
E[X^n] = \begin{cases}
0 & n \text{ odd} \\
\frac{n!}{(2^{n/2} (n/2)!} & n \text{ even}
\end{cases}
$$


\newpage
%%%% 6 %%%%
\item
A random variable $X$ is called log-normal with parameters $\mu$ and $\sigma$ if $X = e^Y$ where
$Y \sim \mathcal{N}(\mu,\sigma^2)$.

\begin{enumerate}
\item
Express the cdf $F_X$ and the density $f_X$ of $X$ in terms of density $\phi$ and cdf $\Phi$ of a
standard normal variable.
\item
What are expectation and variance of $X$?
\item
Let $\mu = 0$ and $\sigma = 1$. Calculate $P[X > 2]$ and find $\alpha$ such that $P[X \leq \alpha] = 99\%$.
\end{enumerate}

Answer:

\begin{enumerate}
\item

\begin{align*}
&F_X(x) = P[X \leq x] =  P[e^Y \leq x] = \begin{cases}
0 & x \leq 0 \\
P[Y \leq \ln x] & x > 0
\end{cases} \\
&Y \sim \mathcal{N}(\mu, \sigma^2) \implies \frac{Y-\mu}{\sigma} \sim \mathcal{N}(0,1) \\
&P[Y \leq \ln x] = P\left[\frac{Y - \mu}{\sigma} \leq \frac{\ln x - \mu}{\sigma}\right] = \Phi\left(\frac{\ln x -\mu}{\sigma}\right) \\
&F_X(x) = \begin{cases}
0 & x \leq 0 \\
\Phi\left(\frac{\ln x -\mu}{\sigma}\right) & x > 0
\end{cases} \\
&f_X(x) = F_X'(x) = \begin{cases}
0 & x \leq 0 \\
\phi\left(\frac{\ln x -\mu}{\sigma}\right) \frac{1}{\sigma x} & x > 0
\end{cases} \\
\end{align*}

\item
\begin{align*}
E[X] &= \int_0^\infty x a \exp \left[ - \left(\frac{\ln x - \mu}{\sqrt{2}\sigma}\right)^2\right] \frac{1}{\sigma x} dx, \quad a =\frac{1}{\sqrt{2\pi}} \\
&= \frac{a}{\sigma}
\int_0^\infty \exp \left[ - \left(\frac{\ln x - \mu}{\sqrt{2}\sigma}\right)^2\right] dx \\
&= a \int_{-\infty}^\infty \exp \left[ \sigma w + \mu  - w^2 / 2\right] dw, \quad w = \frac{\ln x - \mu}{\sigma} \\
&= e^{\mu + \sigma^2 /2} \int_{-\infty}^\infty a e^{-(w - \sigma)^2 / 2} dw \\
&= \boxed{e^{\mu + \sigma^2 /2}} \\
E[X^2] &= \boxed{e^{2\sigma^2  + 2\mu}} \text{ by a similar calculation.}
\end{align*}

\item
\begin{align*}
P[X > 2]  &= 1 - P[X \leq 2] = 1 - \Phi(\ln 2) \approx \boxed{24.24\%} \\
0.99 = P[X \leq \alpha] &= \Phi(\ln \alpha) \\
\ln \alpha &= \Phi^{-1}(0.99) \\
\alpha &= e^{\Phi^{-1}(0.99)} \approx \boxed{10.25}
\end{align*}
\end{enumerate}


\end{enumerate}

\end{document}