\documentclass{article}

\usepackage{amsmath}
\usepackage{amssymb}
\usepackage{amsfonts}
\usepackage{fullpage}
\usepackage{graphicx}
\usepackage{float}


\makeatletter
\def\moverlay{\mathpalette\mov@rlay}
\def\mov@rlay#1#2{\leavevmode\vtop{%
   \baselineskip\z@skip \lineskiplimit-\maxdimen
   \ialign{\hfil$\m@th#1##$\hfil\cr#2\crcr}}}
\newcommand{\charfusion}[3][\mathord]{
    #1{\ifx#1\mathop\vphantom{#2}\fi
        \mathpalette\mov@rlay{#2\cr#3}
      }
    \ifx#1\mathop\expandafter\displaylimits\fi}
\makeatother

\newcommand{\cupdot}{\charfusion[\mathbin]{\cup}{\cdot}}
\newcommand{\bigcupdot}{\charfusion[\mathop]{\bigcup}{\cdot}}

\title{MA 2631 Assignment 7}
\author{Hubert J. Farnsworth}

\setlength\parindent{0pt}
\begin{document}
\maketitle

\begin{enumerate}

%%%% 1 %%%%
\item

Calculate the variance of $X$ and the variance of $Y = 3X+2$ for the random variable X with the probability mass distribution

$$
P [X = 1] = 0.3, \quad P [X = 4] = 0.25, \quad P [X = 7] = 0.4, \quad P [X = 10] = 0.05.
$$

Answer:

\begin{align*}
E[X] &= 4.6\\
E[X^2] &= .3 + 4  + \frac{49 \cdot 2}{5} + 5 = 28.9\\
\text{Var}[X] &= E[X^2] - E[X]^2 = 28.9 - 21.16 = \boxed{7.74}\\
\text{Var}[Y] &= \text{Var}[3X+2] = 3^2 \text{Var}[X] = \boxed{69.66}
\end{align*}

\newpage
%%%% 2 %%%%
\item

Suppose we pick a month at random from a non leap-year calendar and let $X$ be the number of days in that month. Find the mean and the variance of $X$.

Answer:

$$
P[X = 28] = \frac{1}{12}, \quad P[X=30] = \frac{4}{12}, \quad P[X = 31] = \frac{7}{12}.
$$

\begin{align*}
E[X] &= \frac{28 + 120 + 217}{12} = \boxed{\frac{365}{12} = 30.41\overline{6}}\\
E[X^2] &= \frac{28^2 + 30^2\cdot 4 + 31^2 \cdot 7}{12} 
= \frac{784 + 3600 + 6727}{12} = \frac{11,111}{12} = 925.91\overline{6} \\
\text{Var}[X] &= E[X^2] - E[X]^2 = \frac{11,111}{12} - \frac{133,225}{144} = \boxed{\frac{107}{144} = .7430\overline{5}}
\end{align*}

\newpage
%%%% 3 %%%%
\item

Let $Y$ be a binomial distributed random variable with $n$ trials of success probability $p$. Show that $\text{Var}[Y] = np(1-p)$. 

Answer: $Y$ is characterized by $P[Y = k] = \binom{n}{k}p^k(1-p)^{n-k}$ for nonnegative integers $k$ and has mean $E[Y] = np$.

\begin{align*}
E[Y^2] &= \sum_{k=0}^n k^2 P[Y = k] = \sum_{k=0}^n k^2 \binom{n}{k} p^k(1-p)^{n-k} \\
&= \sum_{k=1}^n k^2 \binom{n}{k} p^k(1-p)^{n-k} \\
&= \sum_{k=1}^n kn \binom{n-1}{k-1} p^k(1-p)^{n-k} \quad \text{ identity: }k\binom{n}{k} = n\binom{n-1}{k-1} \\
&= np \sum_{k=1}^n k \binom{n-1}{k-1}p^{k-1}(1-p)^{n-k} \\
&= np \sum_{j=0}^{n-1} (j+1)\binom{n-1}{j} p^j(1-p)^{n-j-1} \\
&= npE[X+1] \quad X \sim \text{Binomial}(n-1,p) \\
&= np(E[X]+ 1) \\
&= np((n-1)p + 1) \\
&\\
\text{Var}[Y] &= E[Y^2] - E[Y]^2 = np((n-1)p + 1) - (np)^2 \\
&= (np)^2 - np^2 + np - (np)^2 \\
&= np(1-p).
\end{align*}



\newpage
%%%% 4 %%%%
\item

Let $Z$ be a geometric distributed random variable with success probability $p$. Calculate $\text{Var}[Z]$. \\

Answer: 
\begin{align*}
P[Z = k] = (1-p)^kp, k = 0,1,2,\dots \\
E[Z] &= \frac{1-p}{p} \\
E[Z^2] &= \sum_{k=0}^\infty k^2(1-p)^kp \\
&= p \sum_{k=1}^\infty ((k-1) + 1)^2(1-p)^k \\
&= p \sum_{k=1}^\infty (k-1)^2(1-p)^k + p\sum_{k=1}^\infty 2(k-1)(1-p)^k + p \sum_{k=1}^\infty (1-p)^k \\
&= (1-p)\sum_{j=0}^\infty j^2p(1-p)^j + 2(1-p)\sum_{j=0}^\infty jp(1-p)^j + (1-p) \\
&= (1-p)E[Z^2] + 2(1-p)E[Z] + (1-p) \\
E[Z^2](1 - (1-p)) &= 2(1-p)\frac{1-p}{p} + 1-p \\
E[Z^2] &= \frac{1-p}{p}\left[\frac{2-2p}{p} + \frac{p}{p}\right] = \frac{(1-p)(2-p)}{p^2} \\
&\\
\text{Var}[Z] &= \frac{(1-p)(2-p)}{p^2} - \left(\frac{1-p}{p}\right)^2  = \frac{1-p}{p^2}(2-p-(1-p))\\
&= \boxed{\frac{1-p}{p^2}}
\end{align*}

\newpage
%%%% 5 %%%%
\item

Assume $X$ is a random variable that only takes on nonnegative integer values and satisfies

$$
P[X \geq n + i \vert X \geq n] = P[X \geq i], \quad n,i \geq 0.
$$

Show that X is a geometric distributed random variable.

Answer: Let $p := P[X = 0]$. \\

\underline{Lemma:} $P[X = n] = pP[X \geq n]$ for any nonnegative integer $n$. \\

Proof. By the assumption, $P[X \geq i] = P[X \geq n+i \vert X \geq n] = P[X \geq n+i] / P[X \geq n]$. In particular for $i = 1$, 

\begin{align*}
1 - P[X = 0] = P[X \geq 1] &= \frac{P[X \geq n+1]}{P[X \geq n]} = 1 - \frac{P[X = n]}{P[X \geq n]} \\
pP[X\geq n] &= P[X = 0]P[X \geq n] = P[X = n]
\end{align*}

\underline{Claim} $P[X = n] = (1-p)^np$, $n = 0,1,2,\dots $ \\

Proof. For the base case $n = 0$, $P[X = 0] = pP[X \geq 0] = p = p(1-p)^0$ using the lemma. Assume the claim holds for some integer $n \geq 0$. 

\begin{align*}
P[X = n+1] &= pP[X \geq n+1] \quad \text{(Lemma)} \\
&= p(P[X\geq n] - P[X =n]) \\
&= p \left( \frac{P[X = n]}{p} - P[X =n] \right) \quad \text{(Lemma)} \\
&= p((1-p)^n - p(1-p)^n) \quad \text{(Inductive Hypothesis)} \\
&= p(1-p)^n(1-p) \\
&= (1-p)^{n+1}p
\end{align*} 

This proves the claim, which shows that $X$ has the probability mass function as a geometrically distributed random variable.


\newpage
%%%% 6 %%%%
\item

The number of errors on a book page follow a Poisson distribution. It has been determined that on 10\% of the pages there is at least one error.

\begin{enumerate}
\item 
Determine the parameter of the Poisson distribution.
\item
What is the expected number of errors on a page?
\end{enumerate}

Answer: $P[X = k] = e^{-\lambda}\frac{\lambda^k}{k!}$, $k = 0,1,2,\dots $

\begin{enumerate}
\item Let $P[X = k]$ be the probability that there are $k$ errors on a page. The given assumption that on 10\% of the pages there is a least one error means $P[X >0] = 0.1$.

$$
e^{-\lambda} \frac{\lambda^0}{0!} = P[X = 0] = P[X \leq 0] = 1- P[X > 0] = 0.9 \implies \lambda = \boxed{\ln 10/9 \approx .10536}
$$

\item

$E[X] = \sum_{k=0}^\infty k e^{-\lambda} \frac{\lambda^k}{k!} = \lambda e^{-\lambda} \sum_{k=1}^\infty \frac{\lambda^{k-1}}{(k-1)!} = \lambda e^{-\lambda} \sum_{j=0}^\infty \frac{\lambda^j}{j!} = \lambda e^{-\lambda} e^{\lambda} = \lambda = \boxed{\ln 10/9 \approx .10536}$
\end{enumerate}

\end{enumerate}

\end{document}